\documentclass[12pt,cspaper]{paper}

\usepackage{url}
\usepackage{color,rotating,epic,subfigure,amssymb,cite,amscd,amsmath,multirow,bm,url,rotating}
\usepackage[margin=2cm]{geometry}

\usepackage{graphicx,epstopdf}
\usepackage{multirow,threeparttable, makecell}
\usepackage{amssymb, siunitx}
\usepackage{amsthm}
\usepackage{epsfig}
\usepackage{cite}
\usepackage{verbatim}
\usepackage{tikz}
\usepackage{longtable}
%\usepackage{gantt}

\usetikzlibrary{arrows,decorations.pathmorphing,backgrounds,fit,positioning,shapes.symbols,chains}
\usetikzlibrary{positioning}
\usepackage{tikz-qtree}
\begin{document}

\title{PEDIA}

\maketitle

\section{Data preprocessing}
We have 348 cases in total. 329 of them are simulated, and 19 are real
cases. There are 9 cases are excluded in preprocessing, because we can
not find the pathogenic mutation gene in their gene list. The
details of cases which we excluded are listed in Table
\ref{table:problem_case}.

After preprocessing, we have 339 cases, and there are 66 cases without Gestalt score in the pathogenic
mutation gene. The detail of these cases are listed in Table
\ref{table:no_gestalt_sample}. Later, we will compare the results
between using 339 cases and 273 cases.

\input{table/problem_sample.tex}
\input{table/sample_detail_no_g}
\section{Results}
\subsection{Comparision the results between with or without Gestalt
score in pathogenic gene}
Table \ref{table:cv} and Table \ref{table:cv_g} show the result of
10-fold cross validation between using all cases and only using the
cases with gestalt score in pathogenic mutation gene.
In Table \ref{table:cv_sample_no_g}, it show the result of 66 cases
without gestalt score in pathogenic mutation gene.
The distribution of pedia score are shown on Table \ref{table:pedia_cv}
and Table \ref{table:pedia_cv_g}.
\input{table/acc_cv_result.tex}
\input{table/cv_sample_without_g.tex}

\input{table/pedia_cv_result.tex}

\subsection{Leave one out cross validation}
The results of LOOCV are in Table \ref{table:loocv} and Table
\ref{table:loocv_g}.
\input{table/acc_loocv_result.tex}
\subsection{Distribution of PEDIA score}
We list the cases which have pedia score below 0 and between 0 and 1 in
Table \ref{table:cv_poor_case} and Table \ref{table:cv_poor_g_case}.
\input{table/pedia_cv_poor_sample.tex}
%\input{table/gene_cv_result.tex}
%\input{table/gene_loocv_result.tex}


\begin{figure}[ht]
  \begin{center}
    \graphicspath{{./table/}}
    \begin{tabular}{cc}
      \includegraphics[width=9 cm, height= 6 cm]{dist_pedia_1KG.png}&
      \includegraphics[width=9 cm, height= 6 cm]{dist_pedia_g_1KG.png}\\
      \includegraphics[width=9 cm, height= 6 cm]{dist_pedia_ExAC.png}&
      \includegraphics[width=9 cm, height= 6 cm]{dist_pedia_g_ExAC.png}\\
      \includegraphics[width=9 cm, height= 6 cm]{dist_pedia_IRAN.png}&
      \includegraphics[width=9 cm, height= 6 cm]{dist_pedia_g_IRAN.png}\\
    \end{tabular}
  \caption{Distribution of pedia score}
  \end{center}
\end{figure}
\begin{figure}[ht]
  \begin{center}
    \graphicspath{{../output/}}
      \includegraphics[width=16 cm, height= 12 cm]{cv/CV_1KG/cv_0/manhattan_all.png}
  \caption{Manhattan plot of all cases in 1KG data set}
  \end{center}
\end{figure}
\begin{figure}[ht]
  \begin{center}
    \graphicspath{{../output/}}
      \includegraphics[width=16 cm, height= 12 cm]{cv_g/CV_1KG/cv_0/manhattan_all.png}
  \caption{Manhattan plot of the cases with gestalt in 1KG data set}
  \end{center}
\end{figure}
\begin{figure}[ht]
  \begin{center}
    \graphicspath{{../output/distribution/CV_1KG/}}
    \begin{tabular}{cc}
      \includegraphics[width=9 cm, height= 6 cm]{dist_gestalt.png}&
      \includegraphics[width=9 cm, height= 6 cm]{dist_FM.png}\\
      \includegraphics[width=9 cm, height= 6 cm]{dist_cadd.png}&
      \includegraphics[width=9 cm, height= 6 cm]{dist_boqa.png}\\
      \multicolumn{2}{c}{\includegraphics[width=9 cm, height= 6 cm]{dist_pheno.png}}\\
    \end{tabular}
  \caption{Distribution of all feature in 1KG data set}
  \end{center}
\end{figure}
\subsection{Influence of excluding each feature}
The feature weights of using all cases are 1.99, 12.43, 6.05, 0.3 and
2.74 for FM, CADD, Gestalt, Boqa and Pheno, and the weights of only
using the cases with gestalt are 4.9, 12.7, 6.86, 1.00 and 1.56.
\input{table/acc_exclude_result.tex}

%\begin{figure}[ht]
%  \graphicspath{./}
%  \includegraphics[width=\textwidth]{rank.png}
%  \caption{Results of training by simulation data and testing on real
%  data. This figure shows cumulative percentage of genes which are predicted to corresponding rank.}
%\end{figure}
%
%\begin{figure}[ht]
%  \graphicspath{./}
%  \includegraphics[width=\textwidth]{rank_cv.png}
%  \caption{Results of 10-fold cross validation. This figure shows cumulative percentage of genes which are predicted to corresponding rank.}
%\end{figure}

%\begin{figure}[ht]
%  \graphicspath{{diff_1KG_cv/}}
%  \includegraphics[width=\textwidth]{Venn-4sets_pa.png}
%  \caption{Venns diagram of feature vector of pathogenic gene.}
%\end{figure}

%\input{diff_result.tex}
%
%\graphicspath{{diff_1KG_cv/}}
%\includegraphics{Venn-4sets_pa.png}

\vfill

\end{document}
